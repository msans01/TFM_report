Figure \ref{fig:gantt} shows the temporal evolution of the various tasks that have constituted the project.
\begin{figure}[H]
  \centering
  \resizebox{\textwidth}{!}{
  
  \begin{ganttchart}[y unit title=0.6cm,
  y unit chart=0.7cm,
  vgrid,hgrid, 
  title label anchor/.style={below=-1.6ex},
  title left shift=.05,
  title right shift=-.05,
  title height=1,
  progress label text={},
  bar height=0.5,
  group right shift=0,
  group top shift=.6,
  group height=.3]{1}{32}
  %labels
  \gantttitle{2023}{16} 
  \gantttitle{2024}{16}\\
  \gantttitle{September}{4} 
  \gantttitle{October}{4} 
  \gantttitle{November}{4} 
  \gantttitle{December}{4} 
  \gantttitle{January}{4} 
  \gantttitle{February}{4} 
  \gantttitle{March}{4}
  \gantttitle{April}{4} \\
  %tasks
  \ganttbar{Literature review}{3}{6} \\
  \ganttbar{Model development}{7}{10} \\
  \ganttbar{Basic solver development}{11}{16} \\
  \ganttbar{GridCal integration}{15}{16} \\
  \ganttbar{Addition of new fatures}{17}{22} \\
  \ganttbar{Benchmarking}{23}{26} \\
  \ganttbar{Documentation}{25}{32} \\
  
  %relations 
  \ganttlink{elem0}{elem1} 
  \ganttlink{elem0}{elem4} 
  \ganttlink{elem1}{elem4} 
  \ganttlink{elem2}{elem4}
  \ganttlink{elem2}{elem3} 
  \ganttlink{elem3}{elem4} 
  \ganttlink{elem4}{elem5} 
  \ganttlink{elem5}{elem6} 
  \end{ganttchart}
  }
  \caption{Gantt Chart of the project.}
  \label{fig:gantt}
\end{figure}
