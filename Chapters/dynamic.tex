
\subsection{Dynamic framework}

The evolution of power systems towards greater complexity and decentralization has introduced new challenges in modeling and analyzing system dynamics. 
Traditional approaches, which often relied on simplifying assumptions and linear models, are increasingly inadequate for capturing the intricate behaviors 
exhibited by modern grids. These grids, characterized by high penetration of Inverter-Based Resources (IBRs), such as photovoltaic and wind generation,
needing more detailed and accurate simulation methods to ensure reliable operation and stability~\cite{LaraTDS}.

Time-domain simulations (TDS) have become a key point in the study of power system dynamics, providing a temporal perspective on system behavior following disturbances.
Unlike steady-state analyses, TDS allows for the examination of transient phenomena, including voltage sags, frequency deviations, and oscillatory modes, 
which are critical for assessing system stability and performance. The ability to simulate the system's response over time enables engineers to design and test control strategies,
protection schemes, and operational protocols in a virtual environment before implementation in the field.

A TDS requires the specification of two parts: the model of the system which must include the differential equations that describe the system and the
integration algorithm to be used to solve those equations over time~\cite{LaraTDS}. 

The system model is a set of dynamic equations that represent and interconnect the components of a system expressed with differential equations. Those
equations are divided between the device equations that store the state of the system and the circuit dynamics of the network. Depending on the type of
simulation to be performed, the model of the system can have many different forms, mainly influenced by simplifications and transformations of the system equations and 
variables. A general explanation of the main methods to simplify and transform systems based on~\cite{LaraTDS} are listed below.

\begin{itemize}
    \item \textit{Averaging dynamics}: These simplifying methods replace fast dynamics of complicated time-varying variables with their average value. 
    An example is setting the system's frequency to its averaged value or replacing discontinuous switching in a converter with its average behavior. 
    Three main methods are distinguished:
    \begin{itemize}
        \item \textit{Steady-state phasors}: Commonly used in engineering using the wave's root-mean-square value instead of the instantaneous value. 
        It requires stationary conditions and homogeneous frequency.
        \item \textit{Dynamic phasors}: Used to describe the behavior of time-varying periodic signals, particularly under conditions where the system frequency 
        or amplitude may vary slowly compared to the fundamental period. These capture both the slowly-varying average and instantaneous phase of the signal. 
        Three different approaches are considered (Fourier approach, time-varying phasors, and Shifted Frequency Analysis (SFA) based), 
        but all of them must ensure that the integral of the signal over the studied period is well-defined and that the signal is band-limited for a given frequency.
        \item \textit{Three-Phase signals \& models}: For three-phase systems, SFA is applied phase by phase to a three-phase \textit{abc} representation to study unbalanced systems. 
        Using the Fortescue transform~\cite{Fortescue}, multi-phase quantities (voltages, currents, electromagnetic fluxes, and impedances) with a common system frequency $w_s$ 
        and offset $\theta$ are represented in a two-dimensional vector dividing the positive and negative-frequency. 
        It should be noted that in balanced systems, the signal phasor is completely determined by the positive-sequence phasor.
    \end{itemize}

    \item \textit{Reference frame transformations}: Such as the Park Transform (or \textit{dq0} transform), which allow the representation of time-variant signals as quasi-static ones, simplifying its analysis.

    \item \textit{Singular Perturbation Theory (SPT)}: Used to analyze dynamic systems with widely separated time scales by decomposing them into fast and slow subsystems. 
    In power systems, SPT is applied to separate the fast transients of synchronous machines and converters (milliseconds) from the slow dynamics of controllers 
    and system frequency (seconds to minutes), enabling more efficient RMS and small-signal analyses.
\end{itemize}


\subsubsection{Types of TDS}

In power systems, time-domain simulations can generally be divided into two main categories~\cite{LaraTDS}. 

The first category, \textit{Quasi-Static Phasor (QSP) simulations}, models the transmission network dynamics in an algebraic manner by considering discrete transitions
between successive steady-state operating points, in other words, QSP focus on dynamics that do not deviate much from steady-state frequency.
These simulations are primarily used to analyze low-frequency phenomena, such as inertial responses and frequency regulation. 
Many different options are exposed for QSP:

\begin{itemize}
    \item \textit{RMS balanced} or \textit{Positive sequence}: The system is modeled using a single phase, the positive sequence. The models are formulated as DAEs of the general form.
     These types of simulators are primarily designed for the analysis of machine angles in balanced transmission systems. 
    \item \textit{RMS unbalanced}: These simulations are applied to networks with asymmetrical configurations. The three sequence networks are simplified according to the type 
    of fault being analyzed. Such models are typically used to assess the system's transient response both during and after unbalanced fault events.
    \item \textit{RMS Dynamic Phasors}: This approach models the network using Fourier dynamic phasors, representing each harmonic component with a subset of differential-algebraic equations. 
    In most RMS applications, the derivative terms of the network are neglected, as they correspond to faster dynamics than those of the connected devices. 
    The model's accuracy depends on the number of harmonics included and on whether the dynamics around each harmonic allow this simplification.
    \item \textit{dq0-Model with Algebraic Network}: This simulation approach applies a dq0 transformation and assumes steady-state conditions by neglecting the derivative terms
     of the network. In balanced systems, it is equivalent to the positive-sequence model. For asymmetric networks or when including harmonics, the method can require larger
     integration steps, and its usefulness is limited due to the resulting time-varying behavior.
\end{itemize}


The second category, \textit{Electromagnetic Transient (EMT) simulations}: These models incorporate detailed system dynamics to capture fast transient phenomena and deviations 
from steady-state conditions. They are commonly used to study high-frequency events such as transmission line dynamics, converter switching, machine flux behavior, 
overvoltages, harmonic propagation, sub-synchronous resonance, transient recovery voltages, and external disturbances like lightning surges. 
A variety of modeling and computational techniques are employed to balance accuracy with the computational cost of these simulations.
Many different options are exposed for EMT:

\begin{itemize}
    \item \textit{Waveform} or \textit{abc-simulations}: They represent the full voltage and current waveforms throughout the simulation, producing a fully time-varying system. 
Waveform simulations require careful initialization and selection of integration techniques, and typically employ detailed transmission line models, such as Bergeron or frequency-dependent
representations, to capture fast electromagnetic phenomena.
    \item \textit{dq0-model}: This approach models the network in a rotating reference frame using a transformed $\pi$-model. In balanced systems, the model produces a time-invariant 
    formulation and shares the same initialization routine as the positive-sequence RMS model. Since the network is represented using only ODEs, the system becomes a set of 
    stiff differential equations due to the multi-rate dynamics, often necessitating simultaneous-implicit solution methods. However, the advantages of this 
    formulation diminish in unbalanced networks or when harmonics are included, as the model then produces time-varying signals similar to full waveform simulations.
    \item \textit{Dynamic Phasors}: This method models power system transients using dynamic phasors, ideal for narrow-band signals around harmonic frequencies. 
    It can produce time-invariant models with additional states or time-variant models for advanced analyses like small-signal sensitivity studies. 
    The approach improves computational efficiency when reducing bandwidth outweighs the cost of extra states.
\end{itemize}

In this thesis, and within the context of the VeraGrid project, the simulations primarily rely on RMS balanced models, which are already implemented in VeraGrid. 
Additionally, EMT waveform  or \textit{abc} domain simulations will be developed as a foundational part of this work.


\subsection{Differential-Algebraic Equations (DAEs)}

At the core of time-domain simulations lie the mathematical models that describe the system's dynamics. These models are typically formulated as Differential-Algebraic Equations
(DAEs), which combine differential equations that represent the system's dynamic behavior with algebraic equations that describe instantaneous relationships between system variables. 
In contrast, Ordinary Differential Equations (ODEs) are used to model systems where all variables change continuously over time without algebraic constraints. 
The choice between DAEs and ODEs depends on the specific characteristics of the system being modeled and the phenomena of interest.

In many dynamic systems, including power systems, their model is not only composed of purely differential equations but also algebraic constraints linking the state
variables. Such systems are naturally represented by differential-algebraic equations (DAEs). Formally, a DAE is an equation of the form: 

\begin{equation}
  F\bigl(\dot{x}(t), x(t), y(t), t\bigr) = 0
\end{equation}


Where $x(t)\in \mathbb{R}^n$ are the differential variables, $y(t)\in \mathbb{R}^m$ are algebraic variables, and the function
 $F$ encodes both the differential and algebraic relations. 
In contrast to an ordinary differential equation (ODE), not all derivatives $\dot{x}$ can be explicitly solved from the system because some variables appear only 
algebraically, that is, without derivative~\cite{CasellaDAE}.

One common formulation is the so-called semi-explicit DAE:  
\begin{equation}
\begin{cases}
\dot x = f\bigl(x,\,y,\,t\bigr) \\
0 = g\bigl(x,\,y,\,t\bigr)
\end{cases}
\end{equation} 

Where $f\colon \mathbb{R}^{n+m+1} \to \mathbb{R}^n$ and $g\colon \mathbb{R}^{n+m+1} \to \mathbb{R}^m$. In this form, one treats some variables $x$ as dynamic and others $y$
as constrained by algebraic equations. The condition to solve the DAE is that the partial jacobian $\partial g / \partial y$ is nonsingular so that $y$ can be locally solved as
a function of $x$ and $t$ \cite{CasellaDAE}. If the partial jacobian is singular an iterative process must be done re-casting the problem until it is possible to solve.  

An alternative is the fully implicit form:  

\begin{equation}
f\bigl(\dot{x},\, x,\, y,\, t \bigr) = 0
\end{equation}

With no explicit separation of differential and algebraic parts. In that representation, one works directly with the combined vector $(\dot{x}, x, y)$ and the solver must treat 
the singularity in $\partial f / \partial \dot{x}$. Implicit DAEs are more general and often arise in multiphysics problems and constrained mechanical systems~\cite{CasellaDAE}.  

 
In power system dynamics, the network and algebraic constraints ( Kirchhoff's current law, network admittance equations, etc.) naturally provide algebraic equations, 
while generator rotor dynamics, excitation systems, governor control, and other dynamic devices supply the differential equations. Thus a typical formulation is  

\begin{equation}
\begin{cases}
\dot{x} = f(x,\, y) \\
0 = g(x,\, y)
\end{cases}
\end{equation}

Where $x$ includes rotor angles, speed deviations, internal voltages, control states, and $y$ includes bus voltages, phase angles, algebraic currents, and so on~\cite{SauerPaiBook}. 
The coupling is strong: every time step, the algebraic network equations must be solved together with the differential updates, typically via Newton-Raphson or other Newton-based 
solvers applied to the full system jacobian.   

This DAE-based representation ensures that the physical constraints of the network are never violated, and that stability and modal analysis reflect the full coupled system behavior,
rather than an oversimplified reduced ODE form.  

\subsection{Symbolic framework}

The development of modern power system simulation tools has progressively incorporated symbolic computation to enhance both the analytical transparency and numerical
robustness of dynamic models. A symbolic framework refers to an environment in which the mathematical equations describing a physical system are manipulated symbolically
before numerical evaluation. This approach enables the explicit representation of the differential and algebraic equations governing power system dynamics, allowing the 
automatic derivation of jacobians and linearized models directly from the symbolic expressions.
Such symbolic processing bridges the gap between manual analytical formulation and automated numerical computation, ensuring consistency and reproducibility across 
different analysis tasks~\cite{SymbolicHantao}.

In general, a symbolic modeling framework treats system equations as algebraic entities rather than arrays of numbers. Each state variable, parameter, and equation
is represented symbolically and stored in structured form. Once the symbolic system is constructed, operations such as differentiation, substitution, and matrix assembly
can be performed analytically. These expressions can then be compiled into efficient numerical routines to be executed by solvers. 
This two-step process, symbolic formulation followed by numerical evaluation, provides the flexibility to modify models, perform analytical checks
and generate adapted code for various simulation modes without redefining the mathematical structure.

Power systems are naturally suited to this formulation, as their behavior is described by a set of coupled DAEs representing both dynamic components and network constraints.
In this context, generators, exciters, governors, and control systems contribute differential equations, whereas network variables, bus voltages, currents, and algebraic 
constraints from Kirchhoff's laws; are represented algebraically. By expressing these components symbolically, they can be automatically assembled into a unified 
system model and linearized analytically to obtain small-signal representations. This methodology ensures that the global model preserves physical consistency and that 
linearization results, such as eigenvalues and participation factors, are computed with exact analytical derivatives rather than numerical approximations.

It should be noted that the DAEs representing each element and device in the network can be simplified or extended depending on the desired level of modelling accuracy. 
A trade-off must therefore be established between computational complexity and numerical accuracy. In practical applications, simplified formulations are often used for 
large-scale steady-state or RMS simulations, whereas more detailed dynamic representations are required for transient and small-signal stability studies. The flexibility 
of the symbolic framework allows the modeller to adapt the formulation of each component accordingly, preserving analytical consistency while optimizing computational 
performance.

\subsection{RMS simulations in VeraGrid}

In the context of VeraGrid, time-domain simulations have been implemented to model the dynamic behavior of power systems.
The framework supports balanced Root Mean Square (RMS) simulations, which are suitable for analyzing low-frequency dynamics in systems dominated by synchronous machines.
The implementation includes:

\begin{itemize}
    \item \textit{Symbolic framework}: A new set of classes is created in order to be able to operate in the symbolic framework
    \item \textit{Models}: Detailed representations of system components, including generators, transformers, and transmission lines, are developed to capture their dynamic
    responses accurately.
    \item \textit{Solver}: The implicit Euler integration method is employed to solve the system's DAEs, ensuring stable and accurate simulation results.
    \item \textit{Initialization}: Procedures are established to determine the steady-state operating point of the system before introducing disturbances,
    providing a consistent starting point for dynamic simulations.
\end{itemize}

This subsection acts as a summary of the already done work on RMS simulations in Veragrid.

\subsubsection{Symbolic framework}
To implement the symbolic formulation in practice, VeraGrid defines a minimal hierarchy  of classes that abstract mathematical entities such as variables,
 constants, and operators. The main classes are stated below.

\begin{itemize}
  \item \textit{Expr}: Expression is an abstract class that can be used as any expression for a node. Depending of it's nature it can work as many different objects.
  \item \textit{Var}: Variables are stored as their name (string) and their value set as Number (integer, float or complex number) or \textit{Expr}. This allows values stored in
  \textit{Var} to be able to change. Some of the variables can be assigned as \textit{Dynamic variable} which means those variables are part of an other device but are also used in
  the DAEs of the device created.
  \item \textit{Const}: Constants are stored as their name (string) and their value set as Number (integer, float or complex number). Those are frozen parameters of the grid.
  \item \textit{BinOp}: Binary operator stores each possibility of binary operation. The following arithmetic operations are defined: addition, subtraction, multiplication, division and exponentiation.
  \item \textit{Func}: Functions store each individual function possible that can be part of an equation: sinus, cosinus, tangent, logarithm, real or imaginary parts among others.
\end{itemize}

All these elements are unified to create \textit{Blocks}, a set of differential and algebraic equations and stated variables that form the symbolic representation of the DAEs of each
element. Blocks also include a set of initializing variables and equations and fixed variables and equations that help initialize the DAEs and allow a fastest solution. All variables in a Block are stored as 
children of that Block in order to allow recursivity. This recursive structure means that each block can contain child elements (either variables or 
other blocks) allowing a hierarchical representation of the power system. Such a design enables complex devices and networks to be constructed from smaller symbolic components 
while maintaining full analytical traceability. Moreover, parameters are also defined as the variables or constants affected by events that can be added to the grid.

\subsubsection{RMS models}

VeraGrid's version of the symbolic framework includes three different types of elements in the grid: buses, injections and branches. Injections include all types of elements
that contribute actively to the system: generators, loads, batteries, shunts... Branches include the devices that connect buses: lines, transformers, converters... 

In this subsection, the RMS models of the grid elements in VeraGrid are exposed.

\paragraph{Injections}

\begin{itemize}
\item \textbf{Static load}
The model for the static load is quite simple since $P$ and $Q$ are meant to be constant. Therefore, the algebraic equations
corresponding to the load are only initializing the power to its value.
\begin{equation}
\begin{cases}
P_l -P_{l,0} = 0\\
Q_l -Q_{l,0} = 0
\end{cases}
\label{eq:alg_eq_load}
\end{equation}

Where:
\begin{itemize}
  \item $P_l$: (Algebraic variable) Active power consumed by the load [MW].
  \item $Q_l$: (Algebraic variable) Reactive power consumed by the load [Mvar].
  \item $P_{l,0}$: (Algebraic variable) Active power consumed by the load at time 0 [MW].
  \item $Q_{l,0}$: (Algebraic variable) Reactive power consumed by the load at time 0 [Mvar].
\end{itemize}

\item  \textbf{Shunt}
For the shunt, the algebraic equations are also simple. They define the formula for the active and reactive power.

\begin{equation}
\begin{cases}
P_{sh} - g V^2 = 0 \\
P_{sh} - b V^2 = 0
\end{cases}
\label{eq:alg_eq_shunt}
\end{equation}

Where:
\begin{itemize}
  \item $P_{sh}$: (Algebraic variable) Active power injected or absorbed by the shunt [MW].
  \item $Q_{sh}$: (Algebraic variable) Reactive power injected or absorbed by the shunt [Mvar].
  \item $g$: (Constant parameter) Shunt conductance [p.u.].
  \item $b$: (Constant parameter) Shunt susceptance [p.u.].
  \item $V$: (Constant parameter) Shunt voltage [p.u.].
\end{itemize}

\item \textbf{Synchronous generator}
The synchronous generator is defined by a set of differential(or state) and algebraic equations. 


\begin{equation}
\begin{cases}
\delta = 2\pi f\,(\omega - \omega_{ref}) \\
\omega = \dfrac{\Gamma_m - \Gamma_e - D(\omega - \omega_{ref})}{M}
\end{cases}
\label{eq:diff_eq_gen}
\end{equation}

\begin{equation}
\begin{cases}
\psi_d - (R \cdot i_q + v_q) = 0 \\
\psi_q + (R\cdot i_d + v_d) = 0\\
0 - ( \psi_d + X \cdot i_d - v_f ) = 0\\
0 - (\psi_q + X \cdot i_q) = 0 \\
v_d - (V \cdot \sin(\delta - \theta)) = 0\\
v_q - (V \cdot \cos(\delta - \theta)) = 0\\
\Gamma_e - ( \psi_d \cdot i_q - \psi_q \cdot i_d ) = 0\\
P_g - (v_d \cdot i_d + v_q \cdot i_q) = 0\\
Q_g - (v_q \cdot i_d - v_d \cdot i_q) = 0 \\
\Gamma_m - (\Gamma_{m0} + K_p(\omega - \omega_{ref}) + K_i \cdot e_t) = 0\\
2\pi f \cdot e_t - \delta = 0
\end{cases}
\label{eq:alg_eq_gen}
\end{equation}

Where:
\begin{itemize}
  \item $\delta$: (State variable) Rotor electrical angle with respect to the synchronous reference frame [rad].
  \item $\omega$: (State variable) Rotor angular speed [rad/s].
  \item $P_g$: (Algebraic variable) Active electrical power delivered by the generator [MW].
  \item $Q_g$: (Algebraic variable) Reactive electrical power delivered by the generator [Mvar].
  \item $v_d$: (Algebraic variable) d-axis stator voltage component in Park reference frame [p.u.].
  \item $v_q$: (Algebraic variable) q-axis stator voltage component in Park reference frame [p.u.].
  \item $i_d$: (Algebraic variable) d-axis stator current component in Park reference frame [p.u.].
  \item $i_q$: (Algebraic variable) q-axis stator current component in Park reference frame [p.u.].
  \item $\psi_d$: (Algebraic variable) d-axis stator flux linkage in Park reference frame [p.u.].
  \item $\psi_q$: (Algebraic variable) q-axis stator flux linkage in Park reference frame[p.u.].
  \item $\Gamma_e$: (Algebraic variable) Electromagnetic torque produced by the generator [p.u.].
  \item $\Gamma_m$: (Algebraic variable) Mechanical torque applied to the rotor [p.u.].
  \item $e_t$: (Algebraic variable) Integral term of the governor controller, representing the accumulated speed error [rad/s].
  \item $V$: (Algebraic variable from the bus) Magnitude of the terminal voltage phasor [p.u.].
  \item $\theta$: (Algebraic variable from the bus) Angle of the terminal voltage phasor with respect to the reference frame [rad].
  \item $\Gamma_{m,0}$: (Fixed variable) Initial mechanical torque reference [p.u.].
  \item $f$: (Constant parameter) Nominal electrical frequency [Hz].
  \item $D$: (Constant parameter) Damping coefficient representing mechanical damping effects [p.u.].
  \item $M$: (Constant parameter) Inertia constant of the rotor [p.u.·s].
  \item $R$: (Constant parameter) Stator winding resistance [p.u.].
  \item $X$: (Constant parameter) Stator synchronous reactance [p.u.].
  \item $v_f$: (Constant parameter) Field excitation voltage [p.u.].
  \item $K_p$: (Constant parameter) Proportional gain of the speed governor control [-].
  \item $K_i$: (Constant parameter) Integral gain of the speed governor control [-].
\end{itemize}
\end{itemize}

\paragraph{Branches}

\begin{itemize}

\item \textbf{Line}
The line algebraic equations are the active and reactive power equations corresponding to the $\pi$ model of lines.
In this case, the RMS model for the lines is also valid for the transformers.

\begin{equation}
\begin{cases}
P_f &= V_{f}^2 \, g - g \, V_{f} V_{t} \cos(\theta_{f} - \theta_{t}) + b \, V_{f} V_{t} \cos\Big(\theta_{f} - \theta_{t} + \frac{\pi}{2}\Big) \\
Q_f &= V_{f}^2 \, \Big(-\frac{b_{sh}}{2} - b \Big) - g \, V_{f} V_{t} \sin(\theta_{f} - \theta_{t}) + b \, V_{f} V_{t} \sin\Big(\theta_{f} - \theta_{t} + \frac{\pi}{2}\Big) \\
P_t &= V_{t}^2 \, g - g \, V_{t} V_{f} \cos(\theta_{t} - \theta_{f}) + b \, V_{t} V_{f} \cos\Big(\theta_{t} - \theta_{f} + \frac{\pi}{2}\Big) \\
Q_t &= V_{t}^2 \, \Big(-\frac{b_{sh}}{2} - b \Big) - g \, V_{t} V_{f} \sin(\theta_{t} - \theta_{f}) + b \, V_{t} V_{f} \sin\Big(\theta_{t} - \theta_{f} + \frac{\pi}{2}\Big)
\end{cases}
\label{eq:alg_eq_line}
\end{equation}

Where:
\begin{itemize}
  \item $P_{f}$: (Algebraic variable) Active power at the beginning of the line (at the bus defined as beginning) [MW].
  \item $Q_{f}$: (Algebraic variable)Reactive power at the beginning of the line (at the bus defined as beginning) [Mvar].
  \item $P_{t}$: (Algebraic variable) Active power at the end of the line (at the bus defined as end) [MW].
  \item $Q_{t}$: (Algebraic variable)Reactive power at the end of the line (at the bus defined as end) [Mvar].
  \item $V_f$: (Algebraic variable) Voltage magnitude at the beginning of the line (at the bus defined as beginning) [p.u.].
  \item $V_t$: (Algebraic variable) Voltage magnitude at the end of the line (at the bus defined as end) [p.u.].
  \item $\theta_f$: (Algebraic variable) Voltage angle at the beginning of the line (at the bus defined as beginning) [rad].
  \item $\theta_t$: (Algebraic variable) Voltage angle at the end of the line (at the bus defined as end) [rad].
  \item $g$: (Constant parameter) Line conductance [p.u.].
  \item $b$: (Constant parameter) Line susceptance [p.u.].
  \item $b_{sh}$: (Constant parameter) Shunt susceptance [p.u.].
\end{itemize}

\end{itemize}

\paragraph{Bus}
The buses store the conservation equations itself. Those are algebraic equations in \ref{eq:alg_eq_bus}. For example in a bus where a 
generator is connected, the power that the generator delivers to the bus must be the same as the power in the line connected to the bus. However, the
variables associated to the bus are the voltage magnitude $V$ and angle $\theta$.
All the variables in the bus algebraic equations are defined as \textit{dynamic variables} since they also appear on other devices (such as 
the generators or lines connected to the bus) so those must be able to be callable from all the device's blocks.

\begin{equation}
\begin{cases}
\sum_i P_{inj} + \sum_i P_{branch} = 0\\
\sum_i Q_{inj} + \sum_i Q_{branch} = 0
\end{cases}
\label{eq:alg_eq_bus}
\end{equation}

Where:
\begin{itemize}
  \item $P_{inj}$: (Algebraic variable) Active power from injections. If the injections are generators the power will be positive and if they are loads, it will be negative [MW].
  \item $Q_{inj}$: (Algebraic variable) Reactive power from injections. If the injections are generators the power will be positive and if they are loads, it will be negative [Mvar].
  \item $P_{branch}$: (Algebraic variable) Active power that is transported from or to the line connected to the bus [MW].
  \item $Q_{branch}$: (Algebraic variable) Reactive power that is transported from or to the line connected to the bus [Mvar].
\end{itemize}

\subsubsection{Integration method: Implicit Euler}

In the numerical integration of differential or differential-algebraic equations (DAEs), Euler methods represent one of the most fundamental approaches
for approximating the time evolution of system variables. They discretize the continuous time domain into a series of steps of size $h$,
and approximate the derivative using finite differences. In RMS power system simulations, which combine slow electromechanical dynamics with algebraic network equations,
Euler methods are often used to understand the basic principles of time integration before moving to more advanced schemes~\cite{Butcher2008}.


The explicit, or forward, Euler method estimates the derivative at the current time step $t_i$ and updates the state variable $y$ according to:
\begin{equation}
    y_{i+1} = y_i + h \left.\frac{\delta y}{\delta t}\right|_{t_i, \, y_i}
    \label{eq:forward_euler}
\end{equation}

This simple expression is obtained by integrating the differential equation
$\dot{y} = f(t, y)$ over the interval $[t_i, t_{i+1}]$ and approximating the derivative by its value at $t_i$. The method is straightforward to implement and
computationally inexpensive, as the right-hand side depends only on known quantities at the current time step. However, it is \textit{conditionally stable},
which means that the time step $h$ must remain below a certain limit for the numerical solution to remain stable~\cite{Butcher2008}. 

For stiff systems, those in which some variables evolve much faster than others, this stability condition becomes extremely restrictive.
In power system models, where fast control loops and electromagnetic effects coexist with slow mechanical dynamics, the forward Euler method becomes inefficient and may
even diverge if the step size is not sufficiently small~\cite{LaraTDS}.

On the contrary, the implicit, or backward, Euler method overcomes these limitations by evaluating the derivative at the next time step $t_{i+1}$:

\begin{equation}
    y_{i+1} = y_i + h \left.\frac{\delta y}{\delta t}\right|_{t_{i+1}, \, y_{i+1}}
    \label{eq:backward_euler}
\end{equation}
This can alternatively be expressed as:
\begin{equation}
    y_i = y_{i+1} - h \left.\frac{\delta y}{\delta t}\right|_{t_{i+1}, \, y_{i+1}}
\end{equation}

Since the derivative depends on the unknown future value $y_{i+1}$, this approach is called implicit. Solving for $y_{i+1}$ requires finding the root of a
nonlinear equation, which is typically done using an iterative method such as Newton-Raphson~\cite{Hindmarsh1995}. Although this increases the computational cost per
time step, the backward Euler method provides unconditional stability for linear systems and is therefore well-suited for stiff problems~\cite{Butcher2008}. 

Intuitively, the method tends to damp fast transients, which makes it ideal for systems dominated by both fast and slow dynamics, as in power systems.
Its numerical damping helps maintain stability even with relatively large time steps, although at the expense of reduced accuracy, since the method is only first order.

Applying the implicit Euler method to DAE systems:

\begin{equation}
    \begin{cases}
        \dfrac{x_{i+1} - x_i}{h} = f(x_{i+1}, y_{i+1}) \\
        0 = g(x_{i+1}, y_{i+1})
    \end{cases}
    \label{eq:dae_implicit_step}
\end{equation}

which forms a set of nonlinear algebraic equations in the unknowns $(x_{i+1}, y_{i+1})$. This system is solved iteratively at each time step using Newton-Raphson,
requiring the evaluation of the Jacobian matrix that couples the differential and algebraic components~\cite{Hindmarsh1995}. 

Despite its simplicity and low order of accuracy, the backward Euler method remains one of the most robust integration schemes for stiff power system DAEs.
Its unconditional stability makes it particularly suitable for RMS simulations involving large time constants or strong coupling between network and control equations~\cite{LaraTDS}.


\subsubsection{Initialization}

coses de pablo i marina

