%INTRODUCTION
\subsection{Objectives}

This thesis is part of the ongoing development of the dynamic simulation framework, 
with a focus on extending its capabilities to include small-signal stability analysis and foundational electromagnetic transient (EMT) modeling. 
The work is carried out within the Veragrid environment, where new modules and methodologies are being implemented. 
The main goal is to equip Veragrid with advanced tools for studying the dynamic behavior of power systems, integrating symbolic modeling, numerical routines,
and graphical interfaces for analysis and visualization.

\subsubsection*{General Objectives}

To develop and validate advanced methodologies for dynamic simulation and stability analysis of power systems by incorporating small-signal stability techniques and foundational EMT modeling, fully integrated into the Veragrid environment.

\subsubsection*{Specific Objectives}

\begin{itemize}
    \item To develop the small-signal analysis module for RMS models, including the formulation and linearization of differential-algebraic equations at the operating point, the computation of eigenvalues and participation factors from the Jacobian matrix, and its integration into Veragrid’s graphical interface.
    \item To implement the foundational components for EMT simulation, modeling transmission lines and system elements in the abc domain, applying discretization techniques such as the Dommel algorithm and alternatives like the 2S-DIRK method, and validating the EMT solver using benchmark systems compared against commercial tools such as PSCAD.
    \item To extend symbolic system formulation to support custom models and control schemes, improving numerical routines, and ensuring consistent initialization of dynamic studies.
    \item To validate the developed methodologies through case studies, continuously comparing results with commercial tools to ensure model reliability and correctness.
\end{itemize}




\subsection{Scope}

This thesis is part of the ongoing development of Veragrid, a leading software platform for power system planning and simulation. 
The work focuses on improving dynamic simulation tools for modern grids, particularly in the context of small-signal stability and electromagnetic transient (EMT) modeling.
Over a nine-month period—from September 2025 to May 2026—the project aims to build essential components that support symbolic formulation, numerical validation,
 and integration with existing simulation environments. 

The first major area of focus is the implementation of small-signal stability analysis using RMS-based state-space models.
This includes the computation of eigenvalues and participation factors, symbolic reduction of system equations,
 and integration of these routines into the VeraGrid graphical interface.
  The goal is to provide researchers and engineers with intuitive and accurate tools for identifying dominant modes and assessing system stability under varying conditions.

The second area involves the development of a foundational EMT solver in the abc domain. This includes modeling transmission lines and components,
 implementing discretization techniques such as the Dommel algorithm and two-stage diagonally implicit Runge-Kutta (2S-DIRK) methods,
  and benchmarking solver performance against commercial tools like PSCAD. Although the EMT module is not intended to be exhaustive,
   it serves as a proof of concept for future expansion and integration.

All development is conducted in Python, with an emphasis on code quality, symbolic computation, and reproducibility.
 The thesis also includes continuous benchmarking and validation using real-world data, including industrial cases.
  Technical supervision is provided by the eRoots team, ensuring alignment with architectural standards and long-term project goals.

\subsection{Structure of the document}

llistar tot jeje

\subsection{State of the art}

sdfsfgsfd

\subsection{Veragrid}

dfdfgassa

\subsection{Previous requirements}

Before starting this thesis, it was necessary to have a solid understanding of power system dynamics, numerical methods for differential equations,
 and programming in Python.

\subsubsection{Dynamic framework in Veragrid}

dfdg

\subsubsection{Symbolic formulation}

ewstrw

\subsubsection{DAE}

dsffg

\newpage